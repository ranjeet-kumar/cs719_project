\documentclass[11pt,twoside]{article}
\addtolength{\textwidth}{0.5in}
\usepackage{epsfig,amsfonts,color}
\usepackage{amsmath}
\bibliographystyle{plain}
\usepackage{amssymb, palatino, geometry,url}
\usepackage{algorithmic}
\usepackage[noresetcount,lined,boxed]{algorithm2e} % ... for algorithms
\usepackage[colorlinks=true,linkcolor=blue,citecolor=blue,urlcolor=blue]{hyperref}
\geometry{letterpaper,
          left       = 0.9in,
          right      = 0.9in,
          top        = 0.9in,
          bottom     = 0.9in}
\linespread{1.2}

\usepackage{fancyhdr}
\usepackage{enumerate}
\usepackage{float}
\usepackage{subcaption}
\usepackage{graphicx}
\usepackage{caption}
\newcommand{\twopart}[4]
{
	\left\{
		\begin{array}{ll}
			#1 & \mbox{ } {\textrm{#2}} \\
			#3 & \mbox{ } {\textrm{#4}}
		\end{array}
	\right.
}

\newcommand{\N}{\mathbb{N}}
\newcommand{\Z}{\mathbb{Z}}
\newcommand{\Q}{\mathbb{Q}}
\newcommand{\R}{\mathbb{R}}
\usepackage{lineno}
%\linenumbers

\begin{document}

\title{Analysis of Multi-Scale Energy Markets using Stochatic Optimization Techniques}

\author{\textbf{\textit{ISyE 719 Course Project}}\\ \\Apoorva Sampat and Ranjeet Kumar\\
 {\small Department of Chemical and Biological Engineering}\\
 {\small \;University of Wisconsin, 1415 Engineering Dr, Madison, WI 53706, USA}}
\date{}
\maketitle

\begin{abstract}
Electricity markets operate at multiple timescales (from hours to milliseconds) to ensure that supply and demands are matched in real time. These markets involve uncertainties because future electricity prices and demands are unknown at the time of decision-making, e.g. energy sale and purchase commitments by generators. We use stochastic programming techniques to study flexibility and economic opportunities provided by a battery in these markets, namely day-ahead (1-hour timescale) and real-time (5-minute timescale) markets. In this work we consider uncertainty only in electricity loads and determine optimal participation strategies using methods like receding horizon scheme, dual dynamic programming and *fullproblem (scenario sampling)*. We also determine bounds on expected policy costs by using perfect information and two-stage approximation (with restriction on states). We compare the costs of participation in exclusively in day-ahead market and both day-ahead and real-time markets. Our results show that market participation only in day-ahead energy markets can *significantly* reduce economic flexibility as compared to participating in both levels of markets.
\end{abstract}


\section{Introduction}
A diverse set of energy systems such as generators, batteries, wind turbines and flywheels can participate in electricity markets. This participation is governed by rules set by ISOs (Independent System Operators) such as California ISO, PJM (Pennsylvania-New Jersey-Maryland) Interconnection and Midcontinent ISO, under whose jurisdiction the market falls. The markets are structured at multiple time levels, namely day ahead (hourly market commitments) and real time markets (commitments ranging from minutes to seconds). In day-ahead markets the electricity is traded in intervals of 1 hour with the prices being constant in each interval of 1 hour and varying with intervals. Real time markets, on the other hand can have varying time scales depending on the ISO operating it. The frequency of energy price variation is different for day-ahead and real time markets (Figure \ref{eprices}). 
\begin{figure}[h!tp]
\centering
\begin{subfigure}[b]{0.32\textwidth} \includegraphics[width=\textwidth]{Figures/damprices.pdf} \caption{Day-ahead market}\label{damprices} \end{subfigure} \hfill
\begin{subfigure}[b]{0.32\textwidth} \includegraphics[width=\textwidth]{Figures/qhmprices.pdf} \caption{Quarter-hourly market}\label{qhmprices} \end{subfigure} \hfill
\begin{subfigure}[b]{0.32\textwidth} \includegraphics[width=\textwidth]{Figures/rtmprices.pdf} \caption{Real-time market}\label{rtmprices}\end{subfigure} \hfill
\caption{Energy prices for one day in the markets in California}\label{eprices}
\end{figure}



Apart from price variations, the introduction of more renewable power sources in the grid such as wind and solar energy, there are greater variations and uncertainties in the net load as well. Since these sources depend on intermittent conditions such as weather, they introduce slow dynamics to the grid. Thus systems with faster dynamic responses such as battery and building systems are becoming increasingly important to balance these fluctuations and provide dynamic flexibility to the power grid. Also, factors such as transmission losses, generation cost and congestion affect the value of products (energy, regulation, spinning , non-spinning reserves) offered at different timescales. These fluctuations being inherently uncertain, determining the optimal participation policy requires analysis using stochastic optimization techniques.
\begin{itemize}
\item \textbf{Day-ahead market :}
The day-ahead market is made up of three market processes that run sequentially. First, the ISO runs a market power mitigation test. Bids that fail the test are revised to predetermined limits. Then the integrated forward market establishes the generation needed to meet forecast demand. And last, the residual unit commitment process designates additional power plants that will be needed for the next day and must be ready to generate electricity. Market prices set are based on bids.

A major component of the market is the full network model, which analyzes the active transmission and generation resources to find the least cost energy to serve demand. The model produces prices that show the cost of producing and delivering energy from individual nodes, or locations on the grid where transmission lines and generation interconnect.

Scheduling coordinators (SCs) are pre-qualified entities authorized to transact in the ISO market. The day-ahead market opens for bids and schedules seven days before and closes the day prior to the trade date. Results are published at 1:00 p.m.

\item \textbf{Real-time market :} 
The real-lime market is a spot market in which utilities can buy power to meet the last few increments of demand not covered in their day ahead schedules. It is also the market that secures energy reserves, held ready and available for ISO use if needed, and the energy needed to regulate transmission line stability.

The market opens at 1:00 p.m. prior to the trading day and closes 75 minutes before the start of the trading hour. The results are published about 45 minutes prior to the start of the trading hour. The real-time market system dispatches power plants every 15 and 5 minutes, although under certain grid conditions the ISO can dispatch for a single 1-minute interval.

\end{itemize}
\section{Problem Definition and Decision-Making Setting}\label{sec:setting}
We consider a rechargeable Li-ion battery with a building that is inter-connected to the power grid for providing electricity services. Electricity services imply that batteries can provide power to or draw power from the grid. In our current setting, we do not consider participation in regulation or other ancillary services. The operator of the power grid (ISO) compensates the battery-owners for the electricity services provided. The goal for the battery-owners is to maximize the revenues generated by providing services to the grid and at the same time meeting the load demands from building. Any unmet load demand from the building is penalized.\\
\begin{figure}[h!]
\begin{center}
\includegraphics[width=4in]{Figures/scenario_tree-crop.pdf} \caption{Scenario tree at the beginning of market participation}\label{scenario_tree}\end{center}
\end{figure}

We divide an annual dataset for electricity loads (available over 5-minute intervals) into 52 subsets of weekly load profiles, i.e 2016 intervals of 5-minute each. This division of data helps in capturing the different load profiles over the weekdays and the weekend. We then use these 52 weekly profiles to generate sample scenarios for our computational experiments (Section \ref{sec:exp}). Since the loads are correlated between time intervals, we formulate a multivariate normal distribution for weekly loads using a mean vector and a covariance matrix. The mean vector (of size 2016 x 1) is the mean of the 52 datasets, but the covariance matrix (of size 2016 x 2016) calculated directly from the 52 datasets cannot be used, since its rank can at most be 52. So we use the Ledoit-Wolf Covariance Estimator \cite{ledoit2004well} (to tackle rank deficiency) for estimating a full rank covariance matrix. We generate 50 samples of load profiles (Figure \ref{fig:loads_scenarios}) for a week using the mean and covariance matrix obtained. 
\begin{figure}[h!]
\begin{center}
\includegraphics[width=4in]{Figures/Plots/fullproblem_stoch/loads_scenarios.pdf} \caption{Load scenarios for a week}\label{fig:loads_scenarios}\end{center}
\end{figure}



Cosidering the price data for a full year to be known (deterministic) we can solve the optimization problem to maximize the annual revenues by operating a battery in the power grid for the whole year. In this work, we aim to solve this deterministic optimization problem with two approaches:
\begin{enumerate}
\item Simultaneous approach: In this approach, we formulate a single giant optimization problem for a full year and solve it to produce the optimal power and regulation capacities that the battery should operate with in each time interval in all the three markets. 
\item Rolling (1 day) horizon approach: In this approach, we formulate an optimization problem for a day and solve it to produce the optimal power and regulation capacities for that day, then take the solutions at the end time-point of that day as the initial point for the next day, then formulate the optimization problem for the next day with those initial values and solve for the second day and sequentially we solve for the 365 days. It is called rolling horizon because in this approach we consider a horizon of 1 day and keep shifting the horizon to to the next day after solving the current day.
\end{enumerate}

At the end, we will compare the two solutions and investigate the advantages and disadvantages of the two approaches.

Understanding the economic incentives provided by generation and load flexibility requires of careful consideration of wholesale electricity market structures and diverse products. Figure \ref{fig:market_control_structure} shows the multiscale control structure currently used to balance the power grid. Resources can participate by buying/selling electrical energy and/or providing ancillary services (regulation, reserves). Figure \ref{fig:energy_AS_prices} shows time-varying prices from the California Independent System Operator (CAISO) for three consecutive days. Energy is transacted at three timescales: in the integrated forward market (day-ahead market with 1-hour intervals), in the fifteen minute market, and through the real-time dispatch process (5-minute intervals). Table \ref{tab:caiso_products} lists the different products transacted at each timescale. Histograms for energy prices at different markets are presented in Figure \ref{fig:CAISO_prices2}. As can be seen, prices are less volatile in the day-ahead market and the average price is higher. In the real-time market (FMM, RTD) prices are frequently negative and occasionally exceed \$150/MWh. Energy systems with fast dynamics (e.g., flywheels, batteries) can exploit these fast price fluctuations.

In the U.S., generators/loads provide the hierarchical market structure with addition flexibility via \emph{regulation} and \emph{reserve} ancillary service products. 

\section{Optimization Model for Building-Battery System}\label{sec:model}
\subsection{Deterministic Setting}\label{subsec:deterministic}
First, we formulate an optimization problem to minimize the cost (or maximize revenue) for the building-battery system operating in the electricity market in a deterministic setting assuming no uncertainty in any of the market and demand quantities. In this formulation, the building-battery system participates in the market at two timescales, hourly and 5-minute intervals. Since the price and load signals are discrete, real-time prices and loads available at 5-minute intervals and day-ahead prices at hourly intervals, we make the zero-order hold assumption, i.e. these time-varying signals are constant within an interval of corresponding duration of their timescale. We also assume that the energy stored in the battery before beginning of market participation is fixed at $E_{0}$. With these assumptions, we develop the optimization problem formulation for the system is described below.
\subsubsection{Sets used in the model}
\begin{center}
$\mathcal{T_R} := \{1,..,n_{\textrm{rtm}}\}$\\$ \mathcal{T_D} :=  \{1,..,n_{\textrm{dam}}\}$
\end{center}
Here, $\mathcal{T_R}$ is the set of time indices corresponding to each 5-minute subintervals in the real-time market within each hour, so $n_{rtm}=12$. $\mathcal{T_D}$ is the set of time indices corresponding to the hourly intervals of the day-ahead market. $n_d$ can be chosen appropriately for any number of hours we plan to schedule the market participation policy for the building-battery system.
\subsubsection{Time discretization used in the model}
A time instance for a day-ahead market variable in the model is represented by only the index $k$ ($k \in \mathcal{T_D}$) which corresponds to the interval between hours $k-1$ and $k$. On the other hand, a time instance for a real-time variable in the model is represented by index $(i,k)$, where $i \in \mathcal{T_R}$ and $ k \in \mathcal{T_D}$. The indices $i$ and $k$ in a time instance $(i,k)$ for a real-time variable correspond to the end of the respective 5-minute interval. For example, the time instance ($5,7$) corresponds to the end of the interval between $4^{\text{th}}$ and $5^{\text{th}}$ 5-minute interval within the $7^\textrm{th}$ hour. 

So, if a variable or time-varying parameter is indexed by only one index $k$ it implies that it varies at hourly intervals and belongs to the day-ahead market, whereas a variable or time-varying parameter indexed by $(i,k)$ is a real-time variable. 
\subsubsection{Parameters in the model:}
\begin{itemize}
\item\textbf{Market parameters:}
\begin{itemize}
\item[\textbullet] $\pi_{k}, \forall k \in \mathcal{T_D}$: Electricity price in day-ahead market in the interval between hour $k-1$ to $k$
\item[\textbullet] $\pi_{i,k}, \forall i \in \mathcal{T_R}, k \in \mathcal{T_D}$: Electricity price in real-time market in the 5-minute interval between $(i-1,k)$ and $(i,k)$ within $k^\text{th}$ hour
\item[\textbullet] $\Delta t$: Time interval (in unit of hours) in the real-time market (which is 5 minutes)
\end{itemize}
\item\textbf{Battery parameters:}
\begin{itemize}
\item[\textbullet] $E_{max} = 0.5$ MWh: Energy storage capacity of the battery
\item[\textbullet] $P_{max} = 0.5$ MW: Maximum discharging or charging rate of the battery
\item[\textbullet] $E_{0} = E_{max}$: Energy stored in the battery before beginning of market participation
\end{itemize}
\item\textbf{Building loads:}
\begin{itemize}
\item[\textbullet] $L_{i,k}, \forall i \in \mathcal{T_R}, k \in \mathcal{T_D}$: Building loads in the 5-minute interval between $(i-1,k)$ and $(i,k)$ within $k^\text{th}$ hour
\end{itemize}
\end{itemize}
\subsubsection{Variables in the model:}
\begin{itemize}
\item \textbf{Decision variables:}
\begin{itemize}
\item[\textbullet] $P_{k}, \forall k \in \mathcal{T_D}$: Power committed by battery in day-ahead market in the interval between hour $k-1$ to $k$
\item[\textbullet] $P_{i,k}, \forall i \in \mathcal{T_R}, k \in \mathcal{T_D}$: Power committed by battery in real-time market in the 5-minute interval between $(i-1,k)$ and $(i,k)$ within $k^\text{th}$ hour
\item[\textbullet] $L_{s_{i,k}}, \forall i \in \mathcal{T_R}, k \in \mathcal{T_D}$: Power supplied by battery to the building in the 5-minute interval between $(i-1,k)$ and $(i,k)$ within $k^\text{th}$ hour 
\end{itemize}
\item \textbf{State variables:}
\begin{itemize}
\item[\textbullet] $E_{i,k}, \forall i \in \mathcal{T_R}, k \in \mathcal{T_D}$: Energy stored in the battery at time $(i,k)$ (at the end of $i^{th}$ 5-minute interval within $k^\text{th}$ hour) 
\end{itemize}
\end{itemize}
\subsubsection{Constraints and objective function}
\begin{itemize}
\item Net discharge (power) from the battery at any time:
\begin{align}
&P_{{net}_{i,k}} = P_{i} + P_{i,k} + L_{s_{i,k}}
\end{align}
\label{eq:Pnet}
\item Balance on the energy stored in the battery at each time
\begin{align}
E_{i,k} =& E_{i-1,k}- P_{{net}_{i,k}}\Delta t_r, \quad \forall i \in \mathcal{T_R}, k \in \mathcal{T_D}
\end{align}
\label{eq:ebalance}
In \ref{eq:ebalance}, $E_{0,k} = E_{0}$ for $k \in \lbrace1\rbrace$ and $E_{0,k} = E_{n_{rtm},k-1}$ for $k \in \mathcal{T_R}\setminus{\lbrace1\rbrace}$. 
\item Bounds on variables:
\begin{subequations}
\begin{align}
0 & \leq E_{i,j,k} \leq E_{max}, \forall i \in \mathcal{T_R}, k \in \mathcal{T_D}\\
-P_{max} & \leq P_{i,k} \leq P_{max}, \forall i \in \mathcal{T_R}, k \in \mathcal{T_D}\\
-P_{max} & \leq P_{k}\phantom{i,} \leq P_{max}, \forall k \in \mathcal{T_D}\\
\end{align}
\label{eq:bounds}
\end{subequations}
\item Objective Function:
\begin{align}
\min \quad -\sum\limits_{i \in \mathcal{T_R}}\sum\limits_{k \in \mathcal{T_R}} \pi_{i,k}P_{i,k}\Delta t_r - \sum\limits_{k \in \mathcal{T_D}}\pi_{k}P_{k}\Delta t_r n_{rtm} + \sum\limits_{i \in \mathcal{T_R}}\sum\limits_{k \in \mathcal{T_R}} \pi_{i,k}(L_{i,k}-L_{s_{i,k}})\Delta t_r
\end{align}
\label{objective}
\end{itemize}

\subsection{Optimization Under Uncertainty}
Now, we consider uncertainty in the building loads and extend the optimization formulation in deterministic setting \ref{subsec:deterministic} to formulate an extensive form of stochastic optimization problem to minimize the expected cost (or maximize expected revenue) for the building-battery system. In the formulation of the problem under uncertainty, we assume all the the market participating conditions are same as in the deterministic setting except for the uncertainty in the loads. The extensive form of the stochastic optimization problem for the system is described below.
\subsubsection{Sets used in the model}
\begin{center}
$\mathcal{T_R} := \{1,..,n_{\textrm{rtm}}\}$\\$ \mathcal{T_D} :=  \{1,..,n_{\textrm{dam}}\}$
\end{center}
Here, $\mathcal{T_R}$ is the set of time indices corresponding to each 5-minute subintervals in the real-time market within each hour, so $n_{rtm}=12$. $\mathcal{T_D}$ is the set of time indices corresponding to the hourly intervals of the day-ahead market. $n_d$ can be chosen appropriately for any number of hours we plan to schedule the market participation policy for the building-battery system.
\subsubsection{Time discretization used in the model}
A time instance for a day-ahead market variable in the model is represented by only the index $k$ ($k \in \mathcal{T_D}$) which corresponds to the interval between hours $k-1$ and $k$. On the other hand, a time instance for a real-time variable in the model is represented by index $(i,k)$, where $i \in \mathcal{T_R}$ and $ k \in \mathcal{T_D}$. The indices $i$ and $k$ in a time instance $(i,k)$ for a real-time variable correspond to the end of the respective 5-minute interval. For example, the time instance ($5,7$) corresponds to the end of the interval between $4^{\text{th}}$ and $5^{\text{th}}$ 5-minute interval within the $7^\textrm{th}$ hour. 

So, if a variable or time-varying parameter is indexed by only one index $k$ it implies that it varies at hourly intervals and belongs to the day-ahead market, whereas a variable or time-varying parameter indexed by $(i,k)$ is a real-time variable. 
\subsubsection{Parameters in the model:}
\begin{itemize}
\item\textbf{Market parameters:}
\begin{itemize}
\item[\textbullet] $\pi_{k}, \forall k \in \mathcal{T_D}$: Electricity price in day-ahead market in the interval between hour $k-1$ to $k$
\item[\textbullet] $\pi_{i,k}, \forall i \in \mathcal{T_R}, k \in \mathcal{T_D}$: Electricity price in real-time market in the 5-minute interval between $(i-1,k)$ and $(i,k)$ within $k^\text{th}$ hour
\item[\textbullet] $\Delta t$: Time interval (in unit of hours) in the real-time market (which is 5 minutes)
\end{itemize}
\item\textbf{Battery parameters:}
\begin{itemize}
\item[\textbullet] $E_{max} = 0.5$ MWh: Energy storage capacity of the battery
\item[\textbullet] $P_{max} = 0.5$ MW: Maximum discharging or charging rate of the battery
\item[\textbullet] $E_{0} = E_{max}$: Energy stored in the battery before beginning of market participation
\end{itemize}
\item\textbf{Building loads:}
\begin{itemize}
\item[\textbullet] $L_{i,k}, \forall i \in \mathcal{T_R}, k \in \mathcal{T_D}$: Building loads in the 5-minute interval between $(i-1,k)$ and $(i,k)$ within $k^\text{th}$ hour
\end{itemize}
\end{itemize}
\subsubsection{Variables in the model:}
\begin{itemize}
\item \textbf{Decision variables:}
\begin{itemize}
\item[\textbullet] $P_{k}, \forall k \in \mathcal{T_D}$: Power committed by battery in day-ahead market in the interval between hour $k-1$ to $k$
\item[\textbullet] $P_{i,k}, \forall i \in \mathcal{T_R}, k \in \mathcal{T_D}$: Power committed by battery in real-time market in the 5-minute interval between $(i-1,k)$ and $(i,k)$ within $k^\text{th}$ hour
\item[\textbullet] $L_{s_{i,k}}, \forall i \in \mathcal{T_R}, k \in \mathcal{T_D}$: Power supplied by battery to the building in the 5-minute interval between $(i-1,k)$ and $(i,k)$ within $k^\text{th}$ hour 
\end{itemize}
\item \textbf{State variables:}
\begin{itemize}
\item[\textbullet] $E_{i,k}, \forall i \in \mathcal{T_R}, k \in \mathcal{T_D}$: Energy stored in the battery at time $(i,k)$ (at the end of $i^{th}$ 5-minute interval within $k^\text{th}$ hour) 
\end{itemize}
\end{itemize}
\subsubsection{Constraints and objective function}
\begin{itemize}
\item Net discharge (power) from the battery at any time:
\begin{align}
&P_{{net}_{i,k}} = P_{i} + P_{i,k} + L_{s_{i,k}}
\end{align}
\label{eq:Pnet}
\item Balance on the energy stored in the battery at each time
\begin{align}
E_{i,k} =& E_{i-1,k}- P_{{net}_{i,k}}\Delta t_r, \quad \forall i \in \mathcal{T_R}, k \in \mathcal{T_D}
\end{align}
\label{eq:ebalance}
In \ref{eq:ebalance}, $E_{0,k} = E_{0}$ for $k \in \lbrace1\rbrace$ and $E_{0,k} = E_{n_{rtm},k-1}$ for $k \in \mathcal{T_R}\setminus{\lbrace1\rbrace}$. 
\item Bounds on variables:
\begin{subequations}
\begin{align}
0 & \leq E_{i,j,k} \leq E_{max}, \forall i \in \mathcal{T_R}, k \in \mathcal{T_D}\\
-P_{max} & \leq P_{i,k} \leq P_{max}, \forall i \in \mathcal{T_R}, k \in \mathcal{T_D}\\
-P_{max} & \leq P_{k}\phantom{i,} \leq P_{max}, \forall k \in \mathcal{T_D}\\
\end{align}
\label{eq:bounds}
\end{subequations}
\item Objective Function:
\begin{align}
\min \quad -\sum\limits_{i \in \mathcal{T_R}}\sum\limits_{k \in \mathcal{T_R}} \pi_{i,k}P_{i,k}\Delta t_r - \sum\limits_{k \in \mathcal{T_D}}\pi_{k}P_{k}\Delta t_r n_{rtm} + \sum\limits_{i \in \mathcal{T_R}}\sum\limits_{k \in \mathcal{T_R}} \pi_{i,k}(L_{i,k}-L_{s_{i,k}})\Delta t_r
\end{align}
\label{objective}
\end{itemize}

\section{Computational Experiments}\label{sec:exp}
List of figures

\subsection{Full Problem}
In this case study, we sample 50 scenario paths (out of the $50^{168}$ possibilities) for a week's time period. We then construct a full model for for this week and use real price signals from 01/01/2015 to 01/07/2015 in California ISO \footnote{http://oasis.caiso.com/mrioasis/logon.do}. For every 1 hour interval in the model, we sample a scenario, and use the load data corresponding to that scenario for the 12 intervals (5 minutes each) in that hour. This helps to reduce the size of scenario tree (from $50^{168 \times 12}$ to $50^{168}$). We solve this model (with 50 sampled paths) 100 times to get a confidence interval on the expected revenue. We then compare this expected revenue (obtained by participating in both real time and day-ahead market) to the cases when the battery participates only in either real time or day-ahead market alone.
\begin{itemize}
\item Supplied and unmet load for a sample path
\item Prtm
\item Pdam
\item SOC 
\item Net discharge
\item revenue from energy : \$ 739.73, \\ from dam: 815.74 \\ from rtm:  -84.76 (loss) \\unmet cost: - 8.75   (i.e. buy from outside when prices are negative)
get's more flexibility to dam participation. 
\item only rtm: total revenue: \$ 70.03 , unmet = -9.10, profit from rtm market =  60.93
\item only dam: total revenue: \$ - 921.61, unmet cost = 76.93, profit edam = -844.68. (needs to buy energy to supply to battery) 
\end{itemize}
\begin{tabular}{|c|c|c|c|c|}
\hline 
Market Participated  & Total Revenue (\$) & Unmet Load Cost (\$) & DAM Revenue  (\$) & RTM Revenue (\$) \\ 
\hline 
DAM + RTM & 739.73 & -8.75 & 815.74 & -84.76 \\ 
\hline 
DAM & -921.61 & 76.93 & -844.68 & N/A \\ 
\hline 
RTM & 70.03 & -9.10 & N/A & 60.93 \\ 
\hline 
None & -10,293.23 & 10,293.23 & • & • \\ 
\hline 
\end{tabular} 


\begin{figure}[h!]
\begin{center}
\includegraphics[width=4in]{Figures/Plots/fullproblem_stoch/cumulative_rev_fp_st.pdf} \caption{Trajectory of cumulative revenues when battery participates in both day-ahead and real-time markets}\label{fig:cumulative_rev_fp_st}\end{center}
\end{figure}
\begin{figure}[h!]
\begin{center}
\includegraphics[width=4in]{Figures/Plots/fullproblem_stoch/netpower_fp_st.pdf} \caption{Net battery discharge when battery participates in both day-ahead and real-time markets}\label{fig:netpower_fp_st}\end{center}
\end{figure}
\begin{figure}[h!]
\begin{center}
\includegraphics[width=4in]{Figures/Plots/fullproblem_stoch/Pdam_fp_st.pdf} \caption{Energy participation policy in day-ahead market when battery participates in both day-ahead and real-time markets}\label{fig:Pdam_fp_st}\end{center}
\end{figure}
\begin{figure}[h!]
\begin{center}
\includegraphics[width=4in]{Figures/Plots/fullproblem_stoch/Prtm_fp_st.pdf} \caption{Energy participation policy in real-time market when battery participates in both day-ahead and real-time markets}\label{fig:Prtm_fp_st}\end{center}
\end{figure}
\begin{figure}[h!]
\begin{center}
\includegraphics[width=4in]
{Figures/Plots/fullproblem_stoch/supp_unmet_fp_st.pdf} \caption{Power supplied to building and unmet load when battery participates in both day-ahead and real-time markets}\label{fig:supp_unmet_fp_st}\end{center}
\end{figure}
\begin{figure}[h!]
\begin{center}
\includegraphics[width=4in]
{Figures/Plots/fullproblem_stoch/soc_fp_st.pdf} \caption{Trajectory of the state of charge of battery when battery participates in both day-ahead and real-time markets}\label{fig:soc_fp_st}\end{center}
\end{figure}
\begin{figure}[h!]
\begin{center}
\includegraphics[width=4in]
{Figures/Plots/onlydam/Pdam_fp_st.pdf} \caption{Energy participation policy when battery participates only in day-ahead market.}\label{fig:Pdam_onlydam}\end{center}
\end{figure}
\begin{figure}[h!]
\begin{center}
\includegraphics[width=4in]
{Figures/Plots/onlydam/cumulative_rev_fp_st.pdf} \caption{Trajectory of cumulative revenue when battery participates only in day-ahead market}\label{fig:cumulative_rev_onlydam}\end{center}
\end{figure}
From figures \ref{fig:Pdam_onlydam} and \ref{fig:cumulative_rev_onlydam}, it is clear that the battery is clearly trying to charge itself (buy energy from grid) in order to meet the building loads as it does not have flexibility of buying energy from the real-time market at lower prices and sell to day-ahead market which usually have much higher prices to make profits. The building loads need to be met in real-time (every 5-minutes) while the battery has the opportunity to buy or sell energy only at every hour. So, the battery is forced to store enough energy to meet the building requirement for an hour by buying energy in hourly intervals from the day-ahead market. Whenever it has slightly excess energy available it tries to sell that energy to the day-ahead market to minimize the overall cost. On the other hand, when it participates in both day-ahead and real-time markets, it has the flexibility to buy from the real-time market and sell to the day-ahead market while saving sufficient energy to meet the building requirement in real-time. This is possible because the real-time electricity prices are slightly lower than the day-ahead market prices on average. The real-time prices also fall below the day-ahead market prices very often (refer figures \ref{fig:Pdam_fp_st} and \ref{fig:Prtm_fp_st}) and the battery because of its fast dynamics can capture these moments by buying from real-time market and selling in day-ahead market to make profits. 

\subsubsection{Upper Bound from Mean Value Problem}


\subsubsection{Estimating Bounds with Perfect Information}
\subsubsection{Estimating Bounds with Two-Stage 	Approximation: Restriction}
\subsection{Receding Horizon Heuristic}
We sample 50 paths upto next 24 hours horizon 
\subsection{Stochastic Dual Dynamic Programming}
\begin{figure}[h!]
\begin{center}
\includegraphics[width=4in]
{Figures/Plots/dualdynamic/bounds.pdf} \caption{Trajectory of the lowerand upper bounds obtained at every iteration of stochastic dual dynamic programming}\label{fig:bounds}\end{center}
\end{figure}
\section{Conclusions}
\section{Future Work}
\subsection{Price Uncertainty}
\subsection{Three Layer Markets}



\bibliography{cs719}

\end{document}
